Definir la función recurisva ocurrenciasElementos que toma
como argumentos dos listas y devuelve una lista de parejas, 
en donde cada pareja contiene en su parte izquierda un elemento 
de la segunda lista y en su parte derechael número de veces que 
aparece dicho elemento en la primera lista. Por ejemplo:

\begin{verbatim}
    > ocurrenciasElementos [1,3,6,2,4,7,3,9,7] [5,2,3]
    [(5,0),(2,1),(3,2)]
\end{verbatim}

Yo llegue a la siguiente solución, que usa una función auxiliar
para contar la cantidad de veces que aparece un elemento en una
lista:
\begin{verbatim}
    -- Ocurrencias.hs
    cuentaElemento :: Int -> [Int] -> Int
    cuentaElemento _ [] = 0
    cuentaElemento x [y]
        | x == y    = 1
        | otherwise = 0
    cuentaElemento x (y:ys)
        | x == y    = 1 + cuentaElemento x ys
        | otherwise = cuentaElemento x ys

    ocurrenciasElementos :: [Int] -> [Int] -> [(Int, Int)]
    ocurrenciasElementos _ [] = []
    ocurrenciasElementos xs (y:ys) = (y, cuentaElemento y xs) : 
                                                    ocurrenciasElementos xs ys
\end{verbatim}

La idea es que la función ocurrenciasElementos recorre la lista
de elementos a contar y por cada elemento llama a la función
cuentaElemento que cuenta la cantidad de veces que aparece el
elemento en la lista. La función cuentaElemento recorre la lista
de elementos y por cada elemento compara si es igual al elemento
a contar, si es así suma 1 al contador y sigue con el resto de la
lista, si no es igual sigue con el resto de la lista. Cuando la
lista esta vacía devuelve el contador.

\vspace{.3cm}