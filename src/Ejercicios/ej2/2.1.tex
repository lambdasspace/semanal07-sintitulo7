\textbf{Explicación del resultado:}

En este código en particular se tienen \textbf{dos resultados} debido a la continuación que se presenta, como se explica a continuación: 

Vemos que la continuación se encuentra en:

\[
(\texttt{let/cc} \, k \, (\texttt{set!} \, c \, k) \, 4)
\]

Aquí, el código continuará con su ejecución con \(c = 4\), es decir:

\begin{verbatim}
    > (+ 1 (+ 2 (+ 3 (+ 4 5))))
\end{verbatim}

Lo cual da como resultado \(15\).

Ahora continuamos y nos encontramos con \((c \, 10)\). En este punto vamos a sustituir \(c\) en el punto donde está la continuación con \(10\) y evaluamos, es decir:

\begin{verbatim}
    > (+ 1 (+ 2 (+ 3 (+ 10 5))))
\end{verbatim}

Lo cual es igual a \(21\). Estos serían los dos resultados de la evaluación después de su ejecución.